%%%%%%%%%%%%%%%%%%%%%%%%%%%%%%%%%%%%%%%%%
% Medium Length Professional CV
% LaTeX Template
% Version 3.0 (December 17, 2022)
%
% This template originates from:
% https://www.LaTeXTemplates.com
%
% Author:
% Vel (vel@latextemplates.com)
%
% Original author:
% Trey Hunner (http://www.treyhunner.com/)
%
% License:
% CC BY-NC-SA 4.0 (https://creativecommons.org/licenses/by-nc-sa/4.0/)
%
%%%%%%%%%%%%%%%%%%%%%%%%%%%%%%%%%%%%%%%%%

%----------------------------------------------------------------------------------------
%----------------------------------------------------------------------------------------

\documentclass[10pt]{resume} % Use the resume class

\usepackage[english, russian]{babel} % Use the EB Garamond font
\usepackage{hyperref}

%------------------------------------------------

\name{Иван Мрасов} % Your name to appear at the top

% You can use the \address command up to 3 times for 3 different addresses or pieces of contact information
% Any new lines (\\) you use in the \address commands will be converted to symbols, so each address will appear as a single line.


\address{
  \href{tel:+79027118460}{+7~(902)~711~8460} \\
  \href{mailto:mrasovir@gmail.com}{mrasovir@gmail.com} \\
  \href{https://github.com/mrariv}{GitHub} \\
  \href{https://ru.linkedin.com/in/ivan-mrasov-593703327}{LinkedIn}
}

%----------------------------------------------------------------------------------------

\begin{document}

\begin{rSection}{Образование}
	
	\textbf{НИУ ВШЭ, Москва} \hfill \textit{2022-2026} \\ 
	Образовательная программа \textbf{«Вычислительные социальные науки»}, 3 курс. \\
 Ключевые курсы: Python, Машинное обучение, Глубинное обучение, Эконометрика, Статистика. \\
 	\textit{Средний балл: 7.6 / 10.}
	
\end{rSection}

%----------------------------------------------------------------------------------------
%	WORK EXPERIENCE SECTION
%----------------------------------------------------------------------------------------

\begin{rSection}{Опыт работы}

%------------------------------------------------
\begin{rSubsection}{Школа ЦПМ}{Сентябрь 2024 - Декабрь 2024}{Преподаватель по теории игр}{Москва}
\item Разработал и провёл \textbf{15 интерактивных занятий} по теории игр для учеников олимпиадного класса.
\item Создал \textbf{библиотеку учебных материалов} с визуализацией основных концепций теории игр.
\item Внедрил \textbf{многоуровневую систему оценки} с еженедельными тестами и персональной обратной связью.
\end{rSubsection}
%------------------------------------------------

	\begin{rSubsection}{IND Architects}{Июнь 2024 - Август 2024}{Ассистент создателя курса «AI в архитектуре»}{Москва}
		\item  Оптимизировал \textbf{prompt engineering} для Midjourney, Stable Diffusion и Llama 2, повысив релевантность генерируемых изображений архитектурным требованиям.
		\item  Разработал \textbf{библиотеку из 50 промптов} с учётом архитектурных стилей и параметров моделей, снизив время подготовки промптов для новых задач до 30 минут (по оценкам студентов).
  \item Автоматизировал \textbf{анализ фидбэка}  от более 40 участников курса с помощью Pandas и NLP.
	\end{rSubsection}
        
         \begin{rSubsection}{ВИКО}{Июнь 2023 - Сентябрь 2023}{Офис-менеджер}{Казань}
		\item Внедрил \textbf{единые Excel-шаблоны} для 20+ рабочих процессов, заменив хаотичные записи.
    \item Упростил контроль задач через \textbf{удобную маркировку статусов} (срочно/в работе/завершено).
    \item Систематизировал \textbf{базу контактов} компании с категоризацией по типам проектов.
	\end{rSubsection}


%------------------------------------------------
\end{rSection}
%----------------------------------------------------------------------------------------
%	TECHNICAL STRENGTHS SECTION
%----------------------------------------------------------------------------------------

\begin{rSection}{Навыки}

	\begin{tabular}{@{} >{\bfseries}l @{\hspace{6ex}} l @{}}
 
		Языки программирования & Python, R, C++, C, RISC-V, Shell \\
		Базы данных & SQL, Excel \\
  Инструменты анализа данных & Pandas, NumPy, Scikit-learn, spaCy, NLTK, PyTorch\\ 
		  Визуализация данных & Matplotlib, Seaborn, Plotly, Tableau, Power BI \\
            Владение языками & Английский C1, Немецкий B2, Французский A2
	\end{tabular}

\end{rSection}

%----------------------------------------------------------------------------------------
%	EXAMPLE SECTION
%----------------------------------------------------------------------------------------

\begin{rSection}{Проекты}

\begin{rSubsection}{Участие в хакатоне VK}{}{}{}
		\item Обучил \textbf{модель классификации} для семантического анализа отзывов со значением точности 92\%.
		\item Применённые навыки: анализ данных, Natural Language Processing (NLP), Python.
	\end{rSubsection}

\begin{rSubsection}{Анализ данных о влиянии комьюта на качество жизни студентов НИУ ВШЭ}{}{}{}
		    \item Разработал \textbf{парсер для сбора данных} о времени комьюта студентов и провёл \textbf{каузальный анализ} влияния комьюта на качество жизни.

  \item  Защитил исследование на конференции \textbf{IFTE 2024} с визуализацией данных и аналитическими выводами.

    \item Применённые навыки: Python, парсинг данных, регрессионный анализ, кластерный анализ.
	\end{rSubsection}


 \begin{rSubsection}{Исследование зависимости между религиозностью и политическими убеждениями}{}{}{}
		    \item Провёл \textbf{сравнительный анализ данных} для изучения взаимосвязи религиозных и политических убеждений общественности в России.
    \item Применённые навыки: регрессионный анализ, написание сложных SQL-запросов, Python.
	\end{rSubsection}
\end{rSection}

%----------------------------------------------------------------------------------------

\end{document}
