%%%%%%%%%%%%%%%%%%%%%%%%%%%%%%%%%%%%%%%%%
% Medium Length Professional CV
% LaTeX Template
% Version 3.0 (December 17, 2022)
%
% This template originates from:
% https://www.LaTeXTemplates.com
%
% Author:
% Vel (vel@latextemplates.com)
%
% Original author:
% Trey Hunner (http://www.treyhunner.com/)
%
% License:
% CC BY-NC-SA 4.0 (https://creativecommons.org/licenses/by-nc-sa/4.0/)
%
%%%%%%%%%%%%%%%%%%%%%%%%%%%%%%%%%%%%%%%%%

%----------------------------------------------------------------------------------------
%----------------------------------------------------------------------------------------

\documentclass[10pt]{resume} % Use the resume class

\usepackage[english, russian]{babel} % Use the EB Garamond font
\usepackage{hyperref}
\usepackage{etoolbox}
\patchcmd{\rSubsection}{\par\vspace{0.5em}}{\par\vspace{0.2em}}{}{}

% \usepackage{parskip}
% \setlength{\parskip}{0.98ex}  

\setlength{\parskip}{5pt}
\setlength{\itemsep}{4pt}   

%------------------------------------------------

\name{Иван Мрасов} % Your name to appear at the top

% You can use the \address command up to 3 times for 3 different addresses or pieces of contact information
% Any new lines (\\) you use in the \address commands will be converted to symbols, so each address will appear as a single line.


\address{
  \href{tel:+79027118460}{+7~(902)~711~8460} \\
  \href{mailto:mrasovir@gmail.com}{mrasovir@gmail.com} \\
  \href{https://github.com/mrariv}{GitHub} \\
  \href{https://ru.linkedin.com/in/ivan-mrasov-593703327}{LinkedIn}
}

%----------------------------------------------------------------------------------------

\begin{document}

\begin{rSection}{Образование}
	
	\textbf{НИУ ВШЭ, Москва} \hfill \textit{2022-2026} \\ 
	Образовательная программа \textbf{«Вычислительные социальные науки»}, 3 курс. \\
 Ключевые курсы: Python, Машинное обучение, Глубинное обучение, Эконометрика, Статистика. \\
 	\textit{Средний балл: 8.2 / 10.}
	
\end{rSection}

%----------------------------------------------------------------------------------------
%	WORK EXPERIENCE SECTION
%----------------------------------------------------------------------------------------

\begin{rSection}{Опыт работы}

%------------------------------------------------
\begin{rSubsection}{Школа ЦПМ}{Сентябрь 2024 - Декабрь 2024}{Преподаватель по теории игр}{Москва}
\item Разработал и провёл \textbf{15 интерактивных занятий} по теории игр для учеников олимпиадного класса.
\item Создал \textbf{библиотеку учебных материалов} с визуализацией основных концепций теории игр.
\item Внедрил \textbf{многоуровневую систему оценки} с еженедельными тестами и персональной обратной связью.
\end{rSubsection}
%------------------------------------------------

	\begin{rSubsection}{IND Architects}{Июнь 2024 - Август 2024}{Ассистент создателя курса «AI в архитектуре»}{Москва}
		\item  Оптимизировал \textbf{prompt engineering} для Midjourney, Stable Diffusion и Llama 2 (через Hugging Face API), повысив релевантность генерируемых изображений архитектурным требованиям.
		\item  Разработал \textbf{библиотеку из 50 промптов} с учётом архитектурных стилей и параметров моделей, снизив время подготовки промптов для новых задач до 30 минут (по оценкам студентов).
  \item Автоматизировал \textbf{анализ фидбэка}  от более 40 участников курса с помощью Pandas и NLP.
	\end{rSubsection}

    
%------------------------------------------------
\end{rSection}
%----------------------------------------------------------------------------------------
%	TECHNICAL STRENGTHS SECTION
%----------------------------------------------------------------------------------------

\begin{rSection}{Навыки}

	\begin{tabular}{@{} >{\bfseries}l @{\hspace{6ex}} l @{}}
 
		Языки программирования & Python, R, C++ \\
		Базы данных & SQL, Excel \\
  Инструменты анализа данных & Pandas, NumPy, Scikit-learn, spaCy, NLTK, PyTorch\\ 
		  Визуализация данных & Matplotlib, Seaborn, Plotly, Tableau, Power BI \\
            Владение языками & Английский C1, Немецкий B2, Французский A2
	\end{tabular}

\end{rSection}

%----------------------------------------------------------------------------------------
%	EXAMPLE SECTION
%----------------------------------------------------------------------------------------

\begin{rSection}{Проекты}

\begin{rSubsection}{Участие в хакатоне VK}{}{}{}
    \item Обучил \textbf{NLP-модель} (Python, BERT) для семантического анализа отзывов со значением \textbf{accuracy 92\%}.
\end{rSubsection}

\begin{rSubsection}{\href{https://github.com/mrariv/asl_recognition}{Модель для распознавания жестов в реальном времени (PyTorch, OpenCV)}}{}{}{}
    \item Разработал CNN-архитектуру для классификации 26 жестов ASL с точностью 96\% на тестовом наборе.
    \item Реализовал пайплайн обработки видео в реальном времени (30 FPS) с использованием OpenCV.
    \item Оптимизировал модель для работы с различными цветами кожи и фонами.
\end{rSubsection}

\begin{rSubsection}{\href{https://github.com/mrariv/commute_project}{Анализ влияния комьюта на качество жизни студентов НИУ ВШЭ}}{}{}{}
    \item Разработал \textbf{парсер} и автоматизировал сбор контактных данных (Selenium + pyautogui) для \textbf{каузального анализа} влияния комьюта на качество жизни.
    \item Защитил исследование на конференции \textbf{IFTE 2024} с визуализацией и аналитическими выводами.
\end{rSubsection}

\begin{rSubsection}{\href{https://github.com/mrariv/model_polarization}{Агентно-ориентированная модель политической поляризации}}{}{}{}
    \item Разработал \textbf{агентную модель} с механизмами идеологической и аффективной форм политической поляризации.
    \item Визуализировал изменение позиций агентов и влияние параметров модели (репликация эффектов из литературы).
\end{rSubsection}

\begin{rSubsection}{\href{https://github.com/mrariv/playlist-manager}{API-сервис для организации плейлистов}}{}{}{}
    \item Реализовал сервис на FastAPI для добавления, пометки и фильтрации треков по жанру и тегам.
    \item Настроил базу данных с помощью SQLAlchemy: реализованы модели треков и избранного, фильтрация по контенту.
    \item Создал HTML-интерфейс с поддержкой добавления треков, лайков и фильтрации через API.
\end{rSubsection}

\end{rSection}
%----------------------------------------------------------------------------------------

\end{document}
